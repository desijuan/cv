\documentclass[12pt, a4paper]{article}

\usepackage[pdftex, top=.4cm, bottom=.4cm, left=.6cm, right=.6cm]{geometry}
\usepackage[utf8]{inputenc}
\usepackage[sfdefault, lf]{carlito}
\usepackage[english]{babel}
\usepackage{xcolor}
\usepackage{multicol}
\setlength\columnsep{12pt}
\usepackage{enumitem}
\usepackage{amssymb} % para \bigstar
\usepackage{hyperref}
\hypersetup{colorlinks=true, urlcolor=teal}
\urlstyle{same}

\setlength{\parindent}{0pt}

% VARS
\newcommand{\email}{desijuan89@gmail.com}
\newcommand{\github}{https://github.com/desijuan}
\newcommand{\linkedin}{https://www.linkedin.com/in/desijuan}
\newcommand{\icono}{$\bigstar\ $}
\newcommand{\linea}{\vspace{-.5em}\rule[-1pt]{.975\columnwidth}{1pt}\par}

% TITULOS
\newcommand{\titulo}[1]{\vspace{1.3em}{\Large\color{teal}\bfseries #1}\par\vspace{.35em}}

%%%%%%%%%%%%%%%%%%%%%%%%%%%%%%%%%%%%%%%%%%%%%%%%%%%%%%%%%%%%
%%%%%%%%%%%%%%%%%%%%%%%%%%%%%%%%%%%%%%%%%%%%%%%%%%%%%%%%%%%%

\begin{document}
\pagestyle{empty}
\thispagestyle{empty}

% HEADER %%%%%%%%%%%%%%%%%%%%%%%%%%%%%%%%%%%%%%%%%%%%%%%%%%%
\rule{\textwidth}{1.5pt}\par
\begin{minipage}[c]{.37\textwidth}
  \vspace{7pt}
  {\Huge\color{teal}\bfseries Juan Desimoni}\par
  \vspace{4pt}
  \url{\github}
  \vspace{6pt}
\end{minipage}
\hfill
\begin{minipage}[c]{.5\textwidth}
  \raggedleft
  Rua Dom José Pereira Alves 61, Niterói, RJ, Brazil\par
  (+55) 21 9 9377 2334\par
  \href{mailto:\email}{\email}
\end{minipage}
\rule{\textwidth}{1.5pt}\par
\vspace{1pt}

%%%%%%%%%%%%%%%%%%%%%%%%%%%%%%%%%%%%%%%%%%%%%%%%%%%%%%%%%%%%
\begin{multicols*}{2}

%%%%%%%%%%%%%%%%%%%%%%%%%%%%%%%%%%%%%%%%%%%%%%%%%%%%%%%%%%%%
\titulo{SOFTWARE DEVELOPER / MATHEMATICS PHD}

Hi! My name is Juan Desimoni. I am a mathematician with a deep passion for computers and software development.

I earned my degree in Mathematics from the Universidad de Buenos Aires (UBA), Argentina, and completed a PhD
in Symplectic Geometry at the Universidade Federal Fluminense (UFF), Brazil, under the supervision of Dr. Matias del Hoyo.

For the past three years, I have been working as a full-time software developer at Codex (São Paulo, Brazil), writing
Kotlin code for a multiplatform application (Windows, Linux, and Android). I collaborated with engineers in the
development of 2 engineering modules: one for calculating the mechanical loading (stress) of poles and another for
evaluating the network using the OpenDSS Electric Power Distribution System Simulator.

In my daily work, I enjoy combining my mathematical expertise with my enthusiasm for programming to solve complex
problems efficiently.
% I am not afraid to dive deep into low-level computing concepts when necessary, whether it's
% optimizing performance, working with system internals, or tackling complex technical challenges head-on.

%%%%%%%%%%%%%%%%%%%%%%%%%%%%%%%%%%%%%%%%%%%%%%%%%%%%%%%%%%%%
\titulo{Programming Skills}

\icono C/C++, Make, CMake, Zig.\par
\icono Java, Kotlin, Maven, Gradle.\par
\icono Databases: SQL and NoSQL.\par
\icono Git, Gitlab CI.\par
\icono Linux sysadmin, Bash scripting.

%%%%%%%%%%%%%%%%%%%%%%%%%%%%%%%%%%%%%%%%%%%%%%%%%%%%%%%%%%%%
\titulo{Work Experience}
\textbf{\icono Codex - Software Developer}. \textit{2022 -- now}.\par
\textbf{\icono Instituto de Matemática e Estatística, UFF. Teaching Assisntant}. \textit{2018 -- 2021}.\par
\textbf{\icono Facultad de Ciencias Exactas y Naturales, UBA. Teaching Assistant}. \textit{2015 -- 2017}.\par
\textbf{\icono Ciclo Básico Común, UBA. Professor}. \textit{2015}.\par
\textbf{\icono Instituto Ballester Deutsche Schule} (high school). \textbf{Professor}. \textit{2016}.

%%%%%%%%%%%%%%%%%%%%%%%%%%%%%%%%%%%%%%%%%%%%%%%%%%%%%%%%%%%%
\titulo{Education}
\textbf{\icono PhD in Mathematics. Universidade Federal Fluminense (UFF)}, Niterói, RJ, Brazil. \textit{2018 -- 2023}.\par
\vspace{1pt}
\textbf{\icono Bs \& Ms in Mathematics. Universidad de Buenos Aires (UBA)}, Buenos Aires, Argentina. \textit{2011 -- 2018}.\par
\vspace{1pt}
\textbf{\icono Weekly meetings with professor Douglas Rodrigues (UFF) in which we study topics in Statistics, Data Science and Machine Learning}. \textit{2021 -- 2022}.

\vfill\null
\columnbreak

%%%%%%%%%%%%%%%%%%%%%%%%%%%%%%%%%%%%%%%%%%%%%%%%%%%%%%%%%%%%
\titulo{Languages}
\textbf{\icono Spanish} native.\par
\textbf{\icono Portuguese} fluent.\par
\textbf{\icono English} advanced.\par
\textbf{\icono German} advanced.

%%%%%%%%%%%%%%%%%%%%%%%%%%%%%%%%%%%%%%%%%%%%%%%%%%%%%%%%%%%%
\titulo{Projects}

\newcommand{\espacio}{.25cm}

The following projects constitute my current portfolio. Most of them are written in \textbf{Zig}, which is my favourite programming language.
For my personal projects, in general I tend to use fewer libraries as possible. This gives me an oportunity to dive deeper into the
concepts and in general results in less bloated apps.

\vspace{\espacio}
\textbf{\color{teal}\large Ledger} \github/ledger \par
App to track shared expenses between groups of people. The frontend is written in \textbf{Elm}, and I use \textbf{Bootstrap} to write less CSS.
The backend is written in \textbf{Zig} and has only two dependencies: \textbf{http.zig} and the \textbf{SQLite C-API}, which I use directly
(without bindings). The frontend and backend communicate by sending HTTP requests with JSON-encoded data.

% \linea %%%%%%
% \icono \textbf{Express} server. Stateless REST API.\par
% \textbf{MongoDB} database, with Mongoose.
%
% \textbf{Repo:} \url{\github/ledger-server}\par
% \textbf{Demo:} \url{https://ledger-server.up.railway.app}
%
% \linea %%%%%
% \icono \textbf{React} client. Responsive with Bootstrap.
%
% \textbf{Repo:} \url{\github/ledger-client}\par
% \textbf{Demo:} \url{https://ledger-client.netlify.app}

\vspace{\espacio}
\textbf{\color{teal}\large Bt} \github/bt\par
Command-line \textbf{BitTorrent} client written in \textbf{Zig} with no external dependencies. Currently, it only works on Linux,
as it directly uses \textbf{io\_uring}. In the future, I plan to use libuv to support other platforms. This project is still
a work in progress.

\vspace{\espacio}
\textbf{\color{teal}\large CSS Parser} \github/zig-css-parser \par
Small CSS parser written in \textbf{Zig} with no external dependencies.

\vspace{\espacio}
\textbf{\color{teal}\large Metronome} \github/metronome \par
Command-line metronome for Linux written in \textbf{Zig}. It uses the Linux \textbf{ALSA C-API} (including "alsa/asoundlib.h")
to play WAV sounds and sigaction (including "signal.h") for the timer.

\vspace{\espacio}
\textbf{\color{teal}\large Flatex} \github/flatex-zig \par
Command-line program to merge multiple LaTeX files into a single one. This project was originally written in \textbf{C} (\github/flatex-zig) and
later rewritten in \textbf{Zig} (\github/flatex). Neither version has external dependencies beyond the C standard library and the Zig standard
library, respectively.

\end{multicols*}

\end{document}
