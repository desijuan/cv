\documentclass[12pt, a4paper]{article}

\usepackage[pdftex, top=1cm, bottom=1cm, left=1.5cm, right=1.5cm]{geometry}
\usepackage[utf8]{inputenc}
\usepackage[sfdefault, lf]{carlito}
\usepackage[spanish, es-nolists]{babel}
\usepackage{xcolor}
\usepackage{multicol}
\setlength\columnsep{12pt}
\usepackage{enumitem}
\usepackage{amssymb} % para \bigstar
\usepackage{hyperref}
\hypersetup{colorlinks=true, urlcolor=teal}
\urlstyle{same}

\setlength{\parindent}{0pt}

% VARS
\newcommand{\email}{desijuan89@gmail.com}
\newcommand{\github}{https://github.com/desijuan}
\newcommand{\linkedin}{https://www.linkedin.com/in/desijuan}
\newcommand{\icono}{$\bigstar\ $}
\newcommand{\linea}{\vspace{-.5em}\rule[-1pt]{.975\columnwidth}{1pt}\par}

% TITULOS
\newcommand{\tit}[1]{\vspace{1.3em}{\Large\color{teal}\bfseries #1}\par\vspace{.35em}}

%%%%%%%%%%%%%%%%%%%%%%%%%%%%%%%%%%%%%%%%%%%%%%%%%%%%%%%%%%%%
%%%%%%%%%%%%%%%%%%%%%%%%%%%%%%%%%%%%%%%%%%%%%%%%%%%%%%%%%%%%

\begin{document}
\pagestyle{empty}
\thispagestyle{empty}

% HEADER %%%%%%%%%%%%%%%%%%%%%%%%%%%%%%%%%%%%%%%%%%%%%%%%%%%
\rule{\textwidth}{1.5pt}\par
\begin{minipage}[c]{.37\textwidth}
  \vspace{7pt}
  {\Huge\color{teal}\bfseries Juan Desimoni}\par
  \vspace{4pt}
  \url{\github}
  \vspace{6pt}
\end{minipage}
\hfill
\begin{minipage}[c]{.5\textwidth}
  \raggedleft
  Rua Dom José Pereira Alves 61, Niterói, RJ, Brasil\par
  (+55) 21 9 9377 2334\par
  \href{mailto:\email}{\email}
\end{minipage}
\rule{\textwidth}{1.5pt}\par
\vspace{1pt}

%%%%%%%%%%%%%%%%%%%%%%%%%%%%%%%%%%%%%%%%%%%%%%%%%%%%%%%%%%%%
\begin{multicols}{2}

%%%%%%%%%%%%%%%%%%%%%%%%%%%%%%%%%%%%%%%%%%%%%%%%%%%%%%%%%%%%
\tit{Objetivo}
\icono Programador Web Full Stack.

%%%%%%%%%%%%%%%%%%%%%%%%%%%%%%%%%%%%%%%%%%%%%%%%%%%%%%%%%%%%
\tit{Habilidades}
\icono Desarrollo Web:
\begin{itemize}[left=0pt, nosep, label={}]
  \item Front-end: \textbf{HTML}, \textbf{CSS}, \textbf{Bootstrap}, \textbf{React}, \textbf{Next.js}.
  \item Back-end: \textbf{Node.js}, \textbf{Express}.
  \item Dev-Ops: \textbf{Docker}.
\end{itemize}
\smallskip

\icono Programación: \textbf{JavaScript}, \textbf{Python}.\par
\icono Bases de Datos: \textbf{SQL}, \textbf{MongoDB}, \textbf{Mongoose}.\par
\icono Control de Versiones: \textbf{Git}, \textbf{GitHub}.\par
\icono Linux: \textbf{Sys admin}, \textbf{Bash}.\par
\icono Estadística y Data Science: \textbf{R}.

%%%%%%%%%%%%%%%%%%%%%%%%%%%%%%%%%%%%%%%%%%%%%%%%%%%%%%%%%%%%
\tit{Proyectos}
\vspace{1pt}
\textbf{\large Ledger}\par
Aplicación web para anotar y dividir gastos entre grupos de personas.
Servidor Express. Base de datos MongoDB.
Cliente React.
Servidor y cliente en plataformas diferentes. Se comunican vía HTTP.

\linea %%%%%%
\icono Servidor \textbf{Express} (Node.js). API REST stateless.\par
Base de datos \textbf{MongoDB} con Mongoose.

\textbf{Repo:} \url{\github/ledger-server}\par
\textbf{Demo:} \url{https://ledger-server.up.railway.app}

\linea %%%%%
\icono Cliente \textbf{React}. Responsivo con \textbf{Bootstrap}.

\textbf{Repo:} \url{\github/ledger-client}\par
\textbf{Demo:} \url{https://ledger-client.netlify.app}

%%%%%%%%%%%%%%%%%%%%%%%%%%%%%%%%%%%%%%%%%%%%%%%%%%%%%%%%%%%%
\tit{Formación Académica}
\textbf{\icono PhD en Matemática. Universidade Federal Fluminense (UFF)}, Niterói, RJ, Brasil. \textit{2018 -- ahora}.\par
\vspace{1pt}
\textbf{\icono Bs \& Ms en Matemática. Universidad de Buenos Aires (UBA)}, Buenos Aires, Argentina. \textit{2011 -- 2018}.\par
\vspace{1pt}
\textbf{\icono Reuniones semanales con el profesor Douglas Rodrigues (UFF), en las que estudiamos temas de Estadística, Data Science y Machine Learning}. \textit{2021 -- ahora}.

%%%%%%%%%%%%%%%%%%%%%%%%%%%%%%%%%%%%%%%%%%%%%%%%%%%%%%%%%%%%
\tit{Idiomas}
\textbf{\icono Español} nativo.\par
\textbf{\icono Portugués} fluente.\par
\textbf{\icono Inglés} avanzado.\par
\textbf{\icono Alemán} avanzado.

\vfill\null
\columnbreak

%%%%%%%%%%%%%%%%%%%%%%%%%%%%%%%%%%%%%%%%%%%%%%%%%%%%%%%%%%%%
\tit{Experiencia Laboral}
\textbf{\icono Instituto de Matemática e Estatística, UFF. Monitor}. \textit{2018 -- 2021}.
\begin{itemize}[left=0pt, nosep, label={}]
  \item 2021-2: Geometria Analítica e Cálculo Vetorial.
  \item 2021-1: Geometria Analítica e Cálculo Vetorial.
  \item 2020-2: Geometria Analítica e Cálculo Vetorial.
  \item 2020-1: Geometria Analítica e Cálculo Vetorial.
  \item 2018-2: Pre-cálculo, Fundamentos de Matemática para Estatística.
  \item 2018-1: Pre-cálculo, Fundamentos de Matemática para Estatística.
\end{itemize}

\textbf{\icono Facultad de Ciencias Exactas y Naturales, UBA. Ayudante}. \textit{2015 -- 2017}.
\begin{itemize}[left=0pt, nosep, label={}]
  \item 2017-1: Geometría Diferencial.
  \item 2016-2: Taller de Cálculo Avanzado.
  \item 2016-1: Análisis Matemático para Biología.
  \item 2015-2: Análisis Matemático 2.
  \item 2015-1: Análisis Matemático para Biología.
\end{itemize}

\textbf{\icono Ciclo Básico Común, UBA. Profesor}. \textit{2015}.
\begin{itemize}[left=0pt, nosep, label={}]
  \item 2015-2: Matemática (51).
\end{itemize}

\textbf{\icono Instituto Ballester Deutsche Schule} (colegio secundario). \textbf{Profesor}. \textit{2016}.
\begin{itemize}[left=0pt, nosep, label={}]
  \item 2016: Matemática 9no año.
\end{itemize}

%%%%%%%%%%%%%%%%%%%%%%%%%%%%%%%%%%%%%%%%%%%%%%%%%%%%%%%%%%%%
\tit{Participaciones en Congresos}
Encontro Conjunto Brasil--Portugal em Matemática, 2022, UFBA, Salvador, BA
$\bullet$ Poisson Conference, 2022, ICMAT, Madrid, España
$\bullet$ Poisson Advanced School, 2022, CRM, Barcelona, España
$\bullet$ Geometry in Algebra and Algebra in Geometry, 2020, UFF, Niterói, RJ
$\bullet$ International Conference on Poisson Geometry, 2019, IMPA, RJ
$\bullet$ School of Poisson Geometry, 2019, IMPA, RJ
$\bullet$ VII Workshop on Poisson Geometry and Related Topics, 2019, UFPR, Curitiba, PR
$\bullet$ 2018 International Congress of Mathematicians, Barra da Tijuca, RJ
$\bullet$ Complex foliations, dynamics and geometry, 2018, UFF, Niterói, RJ
$\bullet$ Geometry at the Frontier III -- School, 2018, UFRO, Temuco, Chile
$\bullet$ Geometry at the Frontier III -- Workshop, 2018, UFRO, Pucón, Chile
$\bullet$ Moduli Spaces in Algebraic Geometry and Applications, 2018, UNICAMP, Campinas, SP
$\bullet$ Summer School on Geometry of Moduli Spaces of Curves (smr3215), 2018, ICTP, Trieste, Italia.
\end{multicols}

\end{document}
